\documentclass[a4paper]{book}
\usepackage[times,inconsolata,hyper]{Rd}
\usepackage{makeidx}
\usepackage[utf8]{inputenc} % @SET ENCODING@
% \usepackage{graphicx} % @USE GRAPHICX@
\makeindex{}
\begin{document}
\chapter*{}
\begin{center}
{\textbf{\huge Package `TREX'}}
\par\bigskip{\large \today}
\end{center}
\begin{description}
\raggedright{}
\inputencoding{utf8}
\item[Title]\AsIs{TRree sap flow EXtractor}
\item[Version]\AsIs{0.0.0.9000}
\item[Description]\AsIs{Performs data assimilation, processing and analyses on sap flow data obtained
with the thermal dissipation method (TDM). The package includes functions for gap filling
time-series data, detecting outliers, calculating data-processing uncertainties and
generating uniform data output and visualisation. The package is designed to deal with
large quantities of data and to apply commonly used data-processing methods. The functions
have been validated on data collected from different tree species across
the northern hemisphere (Peters et al. 2018 <doi: 10.1111/nph.15241>).}
\item[License]\AsIs{MIT + file LICENSE}
\item[LazyData]\AsIs{true}
\item[URL]\AsIs{}\url{https://the-hull.github.io/TREX/}\AsIs{}
\item[BugReports]\AsIs{}\url{https://github.com/the-Hull/TREX/issues}\AsIs{}
\item[Encoding]\AsIs{UTF-8}
\item[Depends]\AsIs{R (>= 3.5.0)}
\item[Imports]\AsIs{chron,
zoo,
solaR,
stats,
graphics,
lubridate,
utils,
grDevices,
lhs,
doParallel,
foreach,
tibble,
dplyr,
sensitivity,
randtoolbox,
boot,
doSNOW,
msm,
parallel,
magrittr}
\item[RoxygenNote]\AsIs{6.1.1}
\item[Suggests]\AsIs{shiny,
DT,
plotly}
\end{description}
\Rdcontents{\R{} topics documented:}
\inputencoding{utf8}
\HeaderA{agg.data}{Aggregation of time-series data}{agg.data}
%
\begin{Description}\relax
Aggregation of time-series data and start/end time selection.
This function provides the option to select the temporal step size
for aggregation of a single time series origination from an \code{\LinkA{is.trex}{is.trex}}-compliant object.
Additionally, the user can define the start and end time of the series and select
the function used for aggregation.
\end{Description}
%
\begin{Usage}
\begin{verbatim}
agg.data(input,
 time.agg = 60*24,
 start = "2012-07-28 00:00",
 end = "2012-10-29 00:00",
 FUN = "mean",
 unit = 60,
 na.rm = TRUE,
 df = FALSE)
\end{verbatim}
\end{Usage}
%
\begin{Arguments}
\begin{ldescription}
\item[\code{input}] An \code{\LinkA{is.trex}{is.trex}}-compliant time series from \code{tdm\_cal.sfd} outputs
(e.g., \code{X\$sfd.mw\$sfd}).

\item[\code{time.agg}] Numeric, the aggregation time in minutes (default = 60).

\item[\code{start}] Character string, the start time for the series. Format has
to be provided in "UTC" (e.g. "2012-05-28 00:00" or Year-Month-Day Hour:Minute).
Starting time should not be earlier than the start of the series.
If not provided the entire series is considered.

\item[\code{end}] character string, the end time for the series.
Format has to be provided in "UTC" (e.g. "2012-06-28 00:50" or Year-Month-Day Hour:Minute).
Starting time should be earlier than the end time and the end time should not exceed the
length of the series.  If not provided the entire series is considered.

\item[\code{FUN}] Quoted function name to compute the summary statistics which can be
applied to all data subsets (see aggregate; including "sum", "mean",
"median", "sd", "se", "min", "max").

\item[\code{unit}] Numeric, the minutes in which a velocity unity is provided
(e.g., \eqn{cm^3 cm^{-2} h^{-1} = 60}{}) for summation (\code{FUN = “sum”}; default = 60).

\item[\code{na.rm}] Logical; if \code{TRUE} missing values are removed (default = \code{TRUE}).

\item[\code{df}] Logical; if \code{TRUE}, output is provided in a \code{data.frame} format
with a timestamp and a value column. If \code{FALSE}, output
is provided as a zoo vector object (default = \code{FALSE}).
\end{ldescription}
\end{Arguments}
%
\begin{Details}\relax
Time series are often derived at variable resolutions.
This function provides the option to aggregate (homogenize) time steps with
standard \code{FUN} statistics. When applying this function to calculate
summed sap flow values (e.g., \eqn{cm^3 cm^{-2} d^{-1}}{}) one needs
to include the velocity unit, as the summation is dependent upon the minimum timestep
of the time series (e.g., \eqn{cm^3 cm^{-2} h^{-1}}{}, \code{unit = 60}).
\end{Details}
%
\begin{Value}
A zoo object or data.frame in the appropriate format for other functionalities.
\end{Value}
%
\begin{Examples}
\begin{ExampleCode}
## Not run: 
#aggregate SFD values to mean hourly and daily sums

raw   <- example.data(type="doy")

input <- is.trex(raw,tz="GMT",time.format="%H:%M",
                  solar.time=TRUE,long.deg=7.7459,ref.add=FALSE,df=FALSE)

input[which(input<0.4)]<-NA

k.input<-tdm_dt.max(dt.steps(input,time.int=15,
               max.gap=180,decimals=10),methods=c("mw"))

sfd.input<-tdm_cal.sfd(k.input,make.plot=FALSE,
                  df=FALSE,wood="Coniferous")$sfd.mw$sfd

# means
output.1hmean <- agg.data(sfd.input,
                       time.agg=60,
                       start="2012-07-28 00:00",
                       end="2012-08-29 00:00",
                       FUN="mean",
                       na.rm=TRUE,
                       df=FALSE)
output.6hmean <- agg.data(sfd.input,
                         time.agg=60*6,
                         start="2012-07-28 00:00",
                         end="2012-08-29 00:00",
                         FUN="mean",
                         na.rm=TRUE,
                         df=FALSE)
plot(output.1hmean,col="cyan")
lines(output.6hmean,col="black")

# daily sums
output.dsum<-agg.data(sfd.input,
                      time.agg=60*24,
                      start="2012-07-28 00:00",
                      end="2012-10-29 00:00",
                      FUN="sum",
                      unit=60,
                      na.rm=TRUE,
                      df=FALSE)
plot(output.dsum)
points(output.dsum,pch=16)

## End(Not run)
\end{ExampleCode}
\end{Examples}
\inputencoding{utf8}
\HeaderA{cal.data}{Calibration Measurements}{cal.data}
\keyword{datasets}{cal.data}
%
\begin{Description}\relax
Returns raw calibration experiment data obtained from literature,
with \emph{K} values combined with gravimetrically determined sap flux density,
as detailed in Flo \emph{et al.} (2019).
The \code{data.frame} contains 22 studies with 37 different species. The data is used
within the \code{\LinkA{tdm\_cal.sfd}{tdm.Rul.cal.sfd}} function to calculate sap flux density.
Description on the genus, species, calibration material, wood porosity and diameter
of the stem is provided in Flo \emph{et al.} (2019).
The presented data is open for public use.
\end{Description}
%
\begin{Usage}
\begin{verbatim}
cal.data
\end{verbatim}
\end{Usage}
%
\begin{Format}
Provides a data.frame with 4024 rows and 10 columns.
\begin{description}

\item[Study] Study from which the data originates (see Flo et al. 2019) (\code{character})
\item[Method] Heat-based sap flow measurement method (TD = Thermal Dissipation) (\code{character})
\item[Genus] Monitored genus (\code{character})
\item[Species] Monitored species (\code{character})
\item[Calibration.material] Description on the calibration method that was used,
including stem segment, whole plant and whole plant without roots (\code{character})
\item[Wood.porosity] Wood structure type of the examined species, including coniferous, diffuse-porous, ring-porous and monocots (\code{character})
\item[Diameter] Diameter at breast height of the calibration subject (in cm) (\code{numeric})
\item[k] Proportional difference between \eqn{\Delta T}{} and \eqn{\Delta T_{max}}{} measured
by the thermal dissipation probes (unitless; \code{numeric})
\item[SFD] Sap flux density measured gravimetrically (in \eqn{cm^3 cm^{-2} h^{-1}}{}; \code{numeric})
\item[Granier] Sap flux density calculated according to Granier et al. 1985 using k
(in \eqn{cm^3 cm^{-2} h^{-1}}{}; using \eqn{43.84 k^{1.231}}{}) (\code{numeric})

\end{description}
\end{Format}
%
\begin{Details}\relax
Currently included studies are given below; individual labels (quoted) can be applied in \code{\LinkA{tdm\_cal.sfd}{tdm.Rul.cal.sfd}} (argument "study"):

\begin{itemize}

\item{} Braun and Schmid 1999
\item{} Cabibel et al. 1991
\item{} Cain 2009
\item{} Chan 2015
\item{} Fuchs et al. 2017
\item{} Granier 1985
\item{} Gutierrez \& Santiago 2005
\item{} Herbst et al. 2007
\item{} Liu et al. 2008
\item{} Lu 2002
\item{} Lu and Chacko 1998
\item{} Oliveira et al. 2006
\item{} Paudel et al. 2013
\item{} Rubilar et al. 2016
\item{} Schmidt-walter et al. 2014
\item{} Sperling et al. 2012
\item{} Sugiura et al. 2009
\item{} Sun et al. 2012
\item{} Vellame et al. 2009
\item{} Hubbard et al. 2010
\item{} Peters et al. 2017
\item{} Steppe et al. 2010


\end{itemize}

\end{Details}
%
\begin{References}\relax
Flo V, Martinez-Vilalta J, Steppe K, Schuldt B, Poyatos, R. 2019.
A synthesis of bias and uncertainty in sap flow methods.
Agricultural and Forest Meteorology 271:362-374. \url{doi: 10.1016/j.agrformet.2019.03.012}

Granier A. 1985. Une nouvelle methode pour la measure du flux de seve brute dans le tronc des arbres.
Annales des Sciences Forestieres 42:193–200. \url{doi: 10.1051/forest:19850204}
\end{References}
\inputencoding{utf8}
\HeaderA{dt.steps}{Determine temporal resolution}{dt.steps}
%
\begin{Description}\relax
Performs minimum time step standardization,
gap filling and start/end time selection. This function
provides the option to select the minimum temporal step size of an
\code{\LinkA{is.trex}{is.trex}} object. Additionally, the user can define the
start and end time of the series and select the minimum size under
which gaps should be filled, using linear interpolation.

Time series have different temporal resolutions.
This function provides the option to standardize the minimum time step by
either performing a linear interpolation when the requested time step
is smaller than the minimum time step of the series or average values when
the requested time step is larger than the minimum time step of the series.
Before this process, the entire time series is converted to a one-minute time
step by applying a linear interpolation (excluding gap \eqn{periods > \code{max.gap}}{}).
\end{Description}
%
\begin{Usage}
\begin{verbatim}
dt.steps(input, start,
       end, time.int = 10, max.gap = 60,
       decimals = 10, df = FALSE)
\end{verbatim}
\end{Usage}
%
\begin{Arguments}
\begin{ldescription}
\item[\code{input}] An \code{\LinkA{is.trex}{is.trex}}-compliant (output) object

\item[\code{start}] Character string providing the start time for the series.
Format has to be provided in “UTC” (e.g., “2012-05-28 00:00” or
Year-Month-Day Hour:Minute). Starting time should not be earlier
than the start of the series.

\item[\code{end}] Character string providing the start time for the series.
Format has to be provided in “UTC” (e.g., “2012-06-28 00:50” or
Year-Month-Day Hour:Minute). End time should be earlier than
the end time and not later than that of the series.

\item[\code{time.int}] Numeric value providing the number of minutes for the
minimum time step. When \code{time.int} is smaller than the minimum time step
of the series, a linear interpolation is applied. If \code{time.int} is
larger than the minimum time step of the series values are average
(after performing a linear interpolation to obtain a one-minute resolution).

\item[\code{max.gap}] Numeric value providing the maximum size of a gap in minutes,
which can be filled by performing a linear interpolation.

\item[\code{decimals}] Integer value defining the number of decimals of the output
(default = 10).

\item[\code{df}] Logical; if \code{TRUE}, output is provided in a \code{data.frame}
format with a timestamp and a value (\eqn{\Delta T}{} or \eqn{\Delta V}{}) column.
If \code{FALSE}, output is provided as a \code{zoo} object (default = FALSE).
\end{ldescription}
\end{Arguments}
%
\begin{Value}
A \code{zoo} object or \code{data.frame} in the appropriate
format for further processing.
\end{Value}
%
\begin{Examples}
\begin{ExampleCode}
input <- is.trex(example.data(type="doy"),
           tz="GMT",time.format="%H:%M", solar.time=TRUE,
           long.deg=7.7459,ref.add=FALSE)
in.ts <- dt.steps(input=input,start='2012-06-28 00:00',end='2012-07-28 00:00',
                   time.int=60,max.gap=120,decimals=6,df=FALSE)
plot(in.ts)
head(in.ts)

\end{ExampleCode}
\end{Examples}
\inputencoding{utf8}
\HeaderA{example.data}{Generate example TDM input data}{example.data}
%
\begin{Description}\relax
This function returns a \code{data.frame} containing standard TDM (thermal dissipation method)
measurements provided in two different formats. The data is obtained from \code{tdm.data}
where \eqn{\Delta V}{} measurements are given for Norway spruce (\emph{Picea abies} Karts.)
growing in a valley in the Swiss Alps. See \code{\LinkA{tdm.data}{tdm.data}} for additional details.
\end{Description}
%
\begin{Usage}
\begin{verbatim}
example.data(type = "timestamp")
\end{verbatim}
\end{Usage}
%
\begin{Arguments}
\begin{ldescription}
\item[\code{type}] Character string, indicating whether the example data should be
displayed with a timestamp (default = \code{“timestamp”})
or separate year, day of day (\code{“doy”}).
\end{ldescription}
\end{Arguments}
%
\begin{Details}\relax
This dataset can be applied for testing the functions provided in \code{TREX}.
\end{Details}
%
\begin{Value}
A \code{data.frame} containing TDM measurements according to a specific type.
\end{Value}
%
\begin{Examples}
\begin{ExampleCode}

# get example data
input_data <- example.data(type = "timestamp")
input_data <- example.data(type = "doy")
head(input_data)

\end{ExampleCode}
\end{Examples}
\inputencoding{utf8}
\HeaderA{gap.fill}{Gap filling by linear interpolation}{gap.fill}
%
\begin{Description}\relax
Fills gaps by linear interpolation between observations.
This function provides the option to define the minimum window under which gaps should
be filled, using linear interpolation.
\end{Description}
%
\begin{Usage}
\begin{verbatim}
gap.fill(input, max.gap = 60, decimals = 10, df = FALSE)
\end{verbatim}
\end{Usage}
%
\begin{Arguments}
\begin{ldescription}
\item[\code{input}] an \code{\LinkA{is.trex}{is.trex}}-compliant object.

\item[\code{max.gap}] Numeric value providing the maximum size of a gap in minutes,
which can be filled by performing a linear interpolation.

\item[\code{decimals}] Integer value defining the number of decimals of the output (default = 10).

\item[\code{df}] Logical; if \code{TRUE}, output is provided in a \code{data.frame}
format with a timestamp and a value (\eqn{\Delta T}{} or \eqn{\Delta V}{}) column.
If \code{FALSE}, output is provided as a \code{zoo} object (default = FALSE).
\end{ldescription}
\end{Arguments}
%
\begin{Value}
A \code{zoo} object or \code{data.frame} in the appropriate format for further processing.
\end{Value}
%
\begin{Examples}
\begin{ExampleCode}
# fill two hour gaps
raw   <- example.data(type = "doy")
input <-
  is.trex(
    raw,
    tz = "GMT",
    time.format = "%H:%M",
    solar.time = TRUE,
    long.deg = 7.7459,
    ref.add = FALSE,
    df = FALSE)

# create gaps in data
input[which(input < 0.4 | input > 0.82)] <- NA
fill_120 <- gap.fill(
  input = input,
  max.gap = 120,
  decimals = 10,
  df = FALSE)
fill_15 <- gap.fill(
  input = input,
  max.gap = 15,
  decimals = 10,
  df = FALSE)
\end{ExampleCode}
\end{Examples}
\inputencoding{utf8}
\HeaderA{is.trex}{Testing and preparing input data}{is.trex}
%
\begin{Description}\relax
Tests if the structure of the input matches the requirements of \code{TREX} functions
and specifies the time zone. The input has to be presented in one of two different \code{data.frame} formats.
i) Timestamp format: including a 1) timestamp of the measurements column (\code{character}), and
2) value of \eqn{\Delta V}{} (or \eqn{\Delta T}{}; [\code{as.numeric}]). ii) DOY format: including a 1) year of measurements column \code{as.integer},
2) day of the year (DOY) of measurement (\code{as.integer}), 3) hour of the measurement (\code{character}), and
4) value of \eqn{\Delta V}{} (or \eqn{\Delta T}{}; [\code{as.numeric}]). TREX functions are applied on time series obtained from a set of
thermal dissipation probes. This includes the option where the thermal dissipation method (TDM) is
used with only a reference and heating probe, or when including addition reference probes (see \code{ref.add}).
These reference probe measurements can be added to the DOY or timestamp format in \eqn{\Delta V}{} (or \eqn{\Delta T}{}) (\code{as.numeric})
labelled \code{ref1}, \code{ref2}, etc. (depending on the number of reference probes). For this function the following
column names have to be present within the \code{data.frame}: "timestamp” or "year” \& "doy” \& "hour” = indicators
of time and "value” = TDM measurements (option "ref1”, "ref2, ..., refn = reference probes).
After specifying the time zone (\code{tz}), one can select whether to standardize the temporal series to solar
time (see \code{solar.time}) by providing the longitude in decimal degrees at which the measurements were
obtained (in \code{long.deg}). All timestamps within the function are rounded to minute resolution and output can
be either provided in a \code{zoo} format (df = \code{FALSE}) or \code{data.frame} (\code{df = TRUE}; default is \code{FALSE}).
\end{Description}
%
\begin{Usage}
\begin{verbatim}
is.trex(data, tz = 'UTC', time.format = '%m/%d/%y %H:%M:%S',
  solar.time = TRUE, long.deg = 7.7459,
   ref.add = FALSE, df = FALSE)
\end{verbatim}
\end{Usage}
%
\begin{Arguments}
\begin{ldescription}
\item[\code{data}] A \code{data.frame} in either timestamp format or doy format.

\item[\code{tz}] Character string, indicates the time zone in which the measurements have been recorded.

\item[\code{time.format}] Character string, indicates the format of the timestamp.

\item[\code{solar.time}] Logical; if \code{TRUE}, time is converted to solar time,
depending upon the location where the measurements have been taken.
If \code{FALSE}, the output is provided in "UTC" (default = \code{FALSE}).

\item[\code{long.deg}] Numeric, longitude in decimal degrees East to perform the solar time
conversion. Only required when \code{solar.time=TRUE}.

\item[\code{ref.add}] Logical; if \code{TRUE}, additional probes provided within data
are considered. The \eqn{\Delta T}{} values are then corrected by subtracting
the \eqn{\Delta T}{} measured between the reference probes from the \eqn{\Delta}{}
T measured between the heated and unheated probe (default = \code{FALSE}).

\item[\code{df}] Logical; if \code{TRUE}, output is provided in a \code{data.frame} format with
a timestamp and a value (\eqn{\Delta T}{} or \eqn{\Delta V}{}) column.
If \code{FALSE}, output is provided as a \code{zoo} object (default = \code{FALSE}).
\end{ldescription}
\end{Arguments}
%
\begin{Details}\relax
To prevent errors occurring in subsequent \code{TREX} functions, it is advised to run this function
for checking the data structure and preparing it for further analyses. For the specific time zone see
\url{https://en.wikipedia.org/wiki/List_of_tz_database_time_zones} or for formatting see \code{\LinkA{OlsonNames}{OlsonNames}()}.
The format of the timestamp has to be provided according to \url{https://www.stat.berkeley.edu/~s133/dates.html}.
For the method behind the solar time conversion, see the solar package (\url{https://cran.r-project.org/web/packages/solaR/}).
The longitude has to be provided in positive decimal degrees for study sites East from the Greenwich meridian and negative for sites to the West.
\end{Details}
%
\begin{Value}
A zoo object or data.frame in the appropriate format for other functionalities.
\end{Value}
%
\begin{Examples}
\begin{ExampleCode}
#validating and structuring example data
raw   <- example.data(type="doy")
input <- is.trex(raw,tz="GMT",time.format="%H:%M",
    solar.time=TRUE,long.deg=7.7459,
    ref.add=FALSE,df=FALSE)
head(raw)
str(input)
head(input)
plot(input)
\end{ExampleCode}
\end{Examples}
\inputencoding{utf8}
\HeaderA{out.data}{Generation of TDM output}{out.data}
%
\begin{Description}\relax
Generates relevant output from the sap flux density (\eqn{SFD}{}) values.
This function provides both \eqn{F_{d}}{} (\eqn{SFD}{} expressed in \eqn{mmol~m^{-2}s^{-1}}{}) and crown conductance values
(\eqn{G_{C}}{}; Meinzer \emph{et al.} 2013, Pappas \emph{et al.} 2018, Peters \emph{et al.} 2018); an analogue to stomatal conductance) in an easily exportable format.
Additionally, the function can perform environmental filtering on \eqn{F_{d}}{} and \eqn{G_{C}}{} and model \eqn{G_{C}}{} sensitivity to vapour pressure deficit (\eqn{VPD}{}).
The user can choose between in- (\code{method = “env.filt”}) or excluding (\code{method = “stat”}) environmental filtering
on the \eqn{G_{C}}{} and adjust the filter threshold manually.
\end{Description}
%
\begin{Usage}
\begin{verbatim}
out.data(input, vpd.input, sr.input, prec.input, peak.hours = c(10:14),
  low.sr = 150, peak.sr = 300, vpd.cutoff = 0.5, prec.lim = 1,
  method = "env.filt", max.quant = 1, make.plot = TRUE)
\end{verbatim}
\end{Usage}
%
\begin{Arguments}
\begin{ldescription}
\item[\code{input}] An \code{\LinkA{is.trex}{is.trex}}-compliant time series from \code{\LinkA{tdm\_cal.sfd}{tdm.Rul.cal.sfd}} outputs
(e.g., \code{X\$sfd.mw\$sfd})

\item[\code{vpd.input}] An \code{\LinkA{is.trex}{is.trex}}-compliant object an individual series of \eqn{VPD}{} in \eqn{kPa}{} (see \code{\LinkA{vpd}{vpd}}).
The extent and temporal resolution should be equal to input.
Use \code{\LinkA{dt.steps}{dt.steps}} to correct if needed.

\item[\code{sr.input}] An \code{\LinkA{is.trex}{is.trex}}-compliant object of an individual series of solar irradiance
(e.g. either PAR or global radiation; see \code{\LinkA{sr}{sr}}).
The extent and temporal resolution should be equal to input. Use \code{\LinkA{dt.steps}{dt.steps}} to correct if needed.
This data is only needed when applying the \code{“env.filt”} method.

\item[\code{prec.input}] An \code{\LinkA{is.trex}{is.trex}}-compliant object of daily precipitation in \eqn{mm~d^{-1}}{} (see \code{\LinkA{preci}{preci}}).
The extent should be equal to input with a daily temporal resolution.
Use \code{\LinkA{dt.steps}{dt.steps}} to correct if needed.
This data is only needed when applying the \code{“env.filt”} method.

\item[\code{peak.hours}] Numeric vector with hours which should be considered as peak-of-the-day hours
(default = \code{c(10:14)}).
This variable is only needed when the \code{“stat”} method is selected.

\item[\code{low.sr}] Numeric threshold value in the unit of the \code{sr.input} time-series (e.g., \eqn{W~m^{-2}}{})
to exclude cloudy days which impact \eqn{G_{C}}{} (default = 150  \eqn{W~m^{-2}}{}).
This variable is only needed when the \code{“env.filt”} method is selected.

\item[\code{peak.sr}] Numeric threshold value in the unit of the sr.input time-series (e.g.,  \eqn{W~m^{-2}}{})
to exclude hours which are not considered as peak-of-the-day hours (default = 300  \eqn{W~m^{-2}}{}).
This variable is only needed when the “\code{env.filt”} method is selected.

\item[\code{vpd.cutoff}] Numeric threshold value in \eqn{kPa}{} for peak-of-the-day mean \eqn{VPD}{} to eliminate unrealistic
and extremely high values of \eqn{G_{C}}{} due to low \eqn{VPD}{} values or high values of \eqn{G_{C}}{} (default = 0.5 \eqn{kPa}{}).

\item[\code{prec.lim}] Numeric threshold value in \eqn{mm~d^{-1}}{}  for daily precipitation to remove rainy days (default = 1 mm d-1).
This variable is only needed when \code{“env.filt”} method is selected.

\item[\code{method}] Character string indicating whether precipitation and solar irradiance data should be used
to determined peak-of-the-day \eqn{G_{C}}{} values and filter the daily \eqn{G_{C}}{} values (“env.filt”)
or not (“stat”; default). When \code{“env.filt”} is selected, \code{input}, \code{vpd.input}, \code{sr.input}, \code{prec.input},
\code{peak.sr}, \code{low.sr}, \code{vpd.cutoff} and \code{prec.lim} have to be provided.
When \code{“stat”} is selected only \code{input}, \code{vpd.input} and \code{peak.hours}.

\item[\code{max.quant}] Numeric, defining the quantile of the \eqn{G_{C}}{} data which should be considered as GC.max (default = 1).

\item[\code{make.plot}] Logical; if \code{TRUE}, a plot is generated presenting the response of \eqn{G_{C}}{} to \eqn{VPD}{}.
\end{ldescription}
\end{Arguments}
%
\begin{Details}\relax
Various relevant outputs can be derived from the \eqn{SFD}{} data.
This function provides the option to recalculate \eqn{SFD}{} to \eqn{F_{d}}{} (expressed in mmol m-2 s-1)
and crown conductance (according to Pappas \emph{et al.} 2018).
\eqn{G_{C}}{} is estimated per unit sapwood area, where \eqn{G_{C} = F_{d} / VPD}{} (in kPa), assuming that
i) the stem hydraulic capacitance between the height of sensor and the leaves is negligible, and
ii) that the canopy is well coupled to the atmosphere. In order to reduce the effect of stem hydraulic capacitance,
peak-of-the-day \eqn{G_{C}}{} are solely considered for calculating daily average \eqn{G_{C}}{}.
Peak-of-the-day conditions are defined by \code{peak.hours} or \code{peak.sr}. Moreover, to analyse the relationship between \eqn{G_{C}}{}
and environmental measurements (e.g., \eqn{VPD}{}), the daily mean peak-of-the-day \eqn{G_{C}}{} values can be restricted to:
i) non-cloudy days (see \code{low.sr}), to reduce the impact of low irradiance on \eqn{G_{C}}{},
ii) non-rainy days (see \code{prec.lim}), as wet leaves are not well coupled to the atmosphere, and
iii) daily mean peak-of-the-day \eqn{G_{C}}{} great then a threshold (see \code{vpd.cutoff}),
to eliminate unrealistically high \eqn{G_{C}}{} values due to low \eqn{F_{d}}{} or \eqn{VPD}{} values (when method = \code{“env.filt”}).
Moreover, the sensitivity of the daily mean peak-of-the-day \eqn{G_{C}}{} to \eqn{VPD}{} is modelled by fitting the following model:

\deqn{G_{C} = \alpha + \beta VPD^{-0.5}}{}

Besides using the raw daily mean peak-of-the-day \eqn{G_{C}}{} values, the function also applies
a normalization where daily mean peak-of-the-day \eqn{G_{C}}{} is standardized to the maximum conductance (GC.max; see \code{max.quant}).
\end{Details}
%
\begin{Value}
A named list of \code{data.frame} objects,
containing the following items:

\begin{description}

\item[raw] A \code{data.frame} containing the input data and filtered values. Columns include the timestamp [,“timestamp”]
(e.g., “2012-01-01 00:00:00”), year of the data [,“year”], day of year [,“doy”], input solar radiance data [,“sr”],
daily average radiance data [,“sr”], input vapour pressure deficit data [,“vpd”], isolated peak-of-the-day vapour pressure
deficit values [,“vpd.filt”], input daily precipitation [,“prec.day”], sap flux density expressed in mmol m-2 s-1 [,“fd”],
crown conductance expressed in \eqn{mmol~m^{-2}~s^{-1}~kPa^{-1}}{} [,“gc”], and the filtered crown conductance [,“gc.filt”]

\item[peak.mean] A \code{data.frame} containing the daily mean crown conductance values.
Columns include the timestamp [,“timestamp”] (e.g., “2012-01-01”), peak-of-the-day vapour pressure deficit [,“vpd.filt”],
the filtered crown conductance \eqn{mmol~m^{-2}~s^{-1}~kPa^{-1}}{} [,“gc.filt”],
and the normalized crown conductance according
to the maximum crown conductance [,“gc.norm”].

\item[sum.mod] A model summary object (see \code{\LinkA{summary}{summary}()})
of the model between \eqn{VPD}{} and \eqn{G_{C}}{}.


\item[sum.mod.norm] A model summary object (see \code{\LinkA{summary}{summary}()})
of the model between \eqn{VPD}{} and \eqn{G_{C}}{}/\eqn{GC.max}{}.




\end{description}

\end{Value}
%
\begin{References}\relax
Meinzer, F. C., D. R. Woodruff, D. M. Eissenstat,
H. S. Lin, T. S. Adams, and K. A. McCulloh. 2013. Above-and belowground controls on water use by
trees of different wood types in an eastern US deciduous forest.
Tree Physiology, 33(4):345–356. \url{doi:10.1093/treephys/tpt012}.

Pappas, C. et al. 2018. Boreal tree hydrodynamics: asynchronous, diverging, yet complementary.
Tree Physiololgy. 38(7):953–964. \url{doi:10.1093/treephys/tpy043}.

Peters, R. L., M. Speich, C. Pappas, A. Kahmen, G. Arx, E. Graf Pannatier,
K. Steppe, K. Treydte, A. Stritih, and P. Fonti. 2018.
Contrasting stomatal sensitivity to temperature and soil drought in mature
alpine conifers, Plant, Cell \& Environment. 42(5):1674-1689. \url{doi:10.1111/pce.13500}.
\end{References}
%
\begin{Examples}
\begin{ExampleCode}
## Not run: 
#Gc response function
#Gc response function
raw   <- is.trex(example.data(type="doy"), tz="GMT",
                 time.format="%H:%M", solar.time=TRUE,
                 long.deg=7.7459, ref.add=FALSE)

input <- dt.steps(input=raw, start="2013-05-01 00:00", end="2013-11-01 00:00",
                   time.int=15, max.gap=60, decimals=10, df=FALSE)

input[which(input<0.2)]<- NA
input <- tdm_dt.max(input, methods=c("dr"), det.pd=TRUE, interpolate=FALSE,
                 max.days=10, df=FALSE)

output.data<- tdm_cal.sfd(input,make.plot=TRUE,df=FALSE,wood="Coniferous")

input<- output.data$sfd.dr$sfd

output<- out.data(input=input, vpd.input=vpd, sr.input=sr, prec.input=preci,
                  low.sr = 150, peak.sr=300, vpd.cutoff= 0.5, prec.lim=1,
                  method="env.filt", max.quant=0.99, make.plot=TRUE)
head(output)


## End(Not run)

\end{ExampleCode}
\end{Examples}
\inputencoding{utf8}
\HeaderA{outlier}{Data cleaning and outlier detection}{outlier}
%
\begin{Description}\relax
This function launches a Shiny application that
(1) visualizes raw and outlier-free time series interactively
(using plotly),
(2) highlights automatically detected outliers,
(3) allows the user to revise the automatically detected outliers
and manually include data points, and
(4) exports the original data and the outlier-free time series in
an \code{\LinkA{is.trex}{is.trex}}-compliant object that can be further processed.
\end{Description}
%
\begin{Usage}
\begin{verbatim}
outlier()
\end{verbatim}
\end{Usage}
%
\begin{Value}
Once the application is launched,
the user can load an \code{.RData} file where a \code{data.frame}
with the timestamp and sensor data are stored.
The timestamp in this \code{data.frame} should be of class \code{Date}.
Many columns can be included (corresponding, for example, to different sensors)
and the user can select through the application the x and y axes of the interactive time series plots.
In addition, the user can provide the units of the imported data
(e.g., degrees \eqn{C}{} or \eqn{mV}{} for \eqn{\Delta T}{} or \eqn{\Delta V}{}, respectively).
A parameter (alpha) for automatic outlier detection can be supplied.
More specifically, the automatic identification of outliers is based on a
two-step procedure:
i) the Tukey’s method (Tukey, 1977) is applied to detect statistical outliers
as values falling outside the range
\eqn{[q_{0.25} - alpha * IQR, q_{0.75} + alpha * IQR]}{},
where \eqn{IQR}{} is the interquartile range
(\eqn{q_{0.75} - q_{0.25}}{})
with \eqn{q_{0.25}}{} denoting the 25\% lower quartile and \eqn{q_{0.75}}{} the
75\% upper quartile, and alpha is a user-defined parameter
(default value \code{alpha = 3};
although visual inspection through the interactive plots allows for adjusting
alpha and optimizing the automatic detection of outliers),
and ii) the first difference (or lag-1 differences) of the raw data are calculated
and data points with lag-1 differences greater
than the mean of the raw input time series, are excluded.

The function does not return a value, but allows the user
to write the raw and outlier-free data in a \code{.RData} file to disk, for
subsequent import via \code{load()}.
\end{Value}
%
\begin{Examples}
\begin{ExampleCode}
## Not run: 
# find example file path
system.file("exdata", "example.RData", package = "TREX", mustWork = TRUE)
# either copy-paste this into the navigation bar of the file selection window
# or navigate here manually for selection

# launch shiny application
outlier()


## End(Not run)

\end{ExampleCode}
\end{Examples}
\inputencoding{utf8}
\HeaderA{preci}{Daily precipitation (raw)}{preci}
\keyword{datasets}{preci}
%
\begin{Description}\relax
Returns an example dataset of daily precipitation data in \eqn{mm~d^{-1}}{}
from 2012-2015 originating from weather stations surrounding the Loetschental in the Swiss Alps.
The data was obtained from the nine nearest weather stations
(6‐to 43‐km distance to the site, including Adelboden, Blatten, Grächen, Montana, Jungfraujoch,
Sion, Ulrichen, Visp, and Zermatt; Federal Office of Meteorology and Climatology MeteoSwiss).
\end{Description}
%
\begin{Usage}
\begin{verbatim}
preci
\end{verbatim}
\end{Usage}
%
\begin{Format}
Provides an \code{\LinkA{is.trex}{is.trex}}-compliant object with 1415 rows and 1 column.

\begin{description}

\item[index] Date of the measurements in solar time (“yyyy-mm-dd”) (\code{character})
\item[value] Daily precipitation (mm d-1) from local weather stations (\code{numeric})

\end{description}
\end{Format}
%
\begin{References}\relax
Peters RL, Speich M, Pappas C, Kahmen A, von Arx G, Graf Pannatier E, Steppe K, Treydte K, Stritih A, Fonti P. 2018.
Contrasting stomatal sensitivity to temperature and soil drought in mature alpine conifers.
Plant, Cell \& Environment 42:1674-1689 \url{doi: 10.1111/pce.13500}
\end{References}
\inputencoding{utf8}
\HeaderA{sr}{Solar radiation measurements (raw)}{sr}
\keyword{datasets}{sr}
%
\begin{Description}\relax
Returns an example dataset of solar irradiance monitoring from 2012-2015 at 1300 m a.s.l. in the Swiss Alps
(Loetschental, Switzerland; Peters et al. 2019). Solar irradiance (\eqn{W m^{-2}}{}) was measured with
15‐min resolution using a microstation (Onset,USA, H21-002 Micro Station)
and pyranometer (Onset, USA, S‐LIB‐M003) positioned in an open field.
\end{Description}
%
\begin{Usage}
\begin{verbatim}
sr
\end{verbatim}
\end{Usage}
%
\begin{Format}
Provides an \code{\LinkA{is.trex}{is.trex}}-compliant object with 135840 rows and 1 column.

\begin{description}

\item[index] Date of the measurements in solar time (“yyyy-mm-dd”) (\code{character})
\item[value] W m-2 values obtained from the site-specific monitoring  (\code{numeric})

\end{description}
\end{Format}
%
\begin{References}\relax
Peters RL, Speich M, Pappas C, Kahmen A, von Arx G, Graf Pannatier E, Steppe K, Treydte K, Stritih A, Fonti P. 2018.
Contrasting stomatal sensitivity to temperature and soil drought in mature alpine conifers.
Plant, Cell \& Environment 42:1674-1689 \url{doi: 10.1111/pce.13500}
\end{References}
\inputencoding{utf8}
\HeaderA{tdm.data}{Sap flow measurements}{tdm.data}
\keyword{datasets}{tdm.data}
%
\begin{Description}\relax
Returns an example thermal dissipation probe (TDM)
dataset with time stamp (n = 1) and doy-columns (n = 3), a value and a species column.
TDM \eqn{\Delta V}{} measurements are provided at a 15-minute resolution from 2012-2015 from
a Norway spruce (\emph{Picea abies} (L.) Karts.) growing at 1300 m a.s.l.
in the Swiss Alps (Loetschental, Switzerland; see Peters \emph{et al.} 2019).
The presented data is open for public use.
\end{Description}
%
\begin{Usage}
\begin{verbatim}
tdm.data
\end{verbatim}
\end{Usage}
%
\begin{Format}
Provides a data.frame with 11,6466  rows and 6 columns.
\begin{description}

\item[timestamp ] Date and time of the measurements (\code{character})
\item[year] Year of measurements (\code{integer})
\item[doy] Day of year (\code{integer})
\item[hour] Hour of the measurements (\code{character})
\item[value] \eqn{\Delta V}{} values obtained from TDM measurements (\code{numeric})
\item[species] Monitored species (\code{character})

\end{description}
\end{Format}
\inputencoding{utf8}
\HeaderA{tdm\_cal.sfd}{Calculate sap flux density}{tdm.Rul.cal.sfd}
%
\begin{Description}\relax
The acquired \eqn{K}{} values are calculated to sap flux density
(\eqn{SFD}{} in \eqn{cm^3 cm^{-2} h^{-1}}{}). As many calibration curves exist
(see Peters \emph{et al}. 2018; Flo \emph{et al.} 2019), the function provides the option to
calculate \eqn{SFD}{} using calibration experiment data from the meta-analyses by
Flo \emph{et al.} (2019; see \code{\LinkA{cal.data}{cal.data}}). Additionally,
raw calibration data can be provided or parameters \eqn{a}{} and \eqn{b}{}
for a specific calibration function (\eqn{aK^b}{}) can be provided.
The algorithm determines for each calibration experiment dataset
the calibration curve (\eqn{SFD = aK^b}{}) and calculates \eqn{SFD}{} from
either the mean of all curves and the 95\% confidence interval
of either all curves, or bootstrapped resampled uncertainty around
the raw calibration experiment data when one calibration dataset is selected.
\end{Description}
%
\begin{Usage}
\begin{verbatim}
tdm_cal.sfd(input, genus, species, study, wood, calib, a, b, decimals,
  make.plot = TRUE, df = FALSE)
\end{verbatim}
\end{Usage}
%
\begin{Arguments}
\begin{ldescription}
\item[\code{input}] An \code{\LinkA{is.trex}{is.trex}}-compliant object (\code{zoo} vector,
\code{data.frame}) of \eqn{K}{} values containing a timestamp and a value column.

\item[\code{genus}] Optional, character vector specifying genus-specific calibration
data that should be used (e.g., \code{c("Picea", "Larix")}). See \code{\LinkA{cal.data}{cal.data}}
for the specific labels (default = Use all).

\item[\code{species}] Optional, character vector of species specific calibration data that should be used,
e.g. \code{c("Picea abies")}. See \code{\LinkA{cal.data}{cal.data}} for the specific labels (default = Use all).

\item[\code{study}] Optional character vector of study specific calibration data that
should be used (e.g., \code{c("Peters et al. 2018"}) ). See \code{\LinkA{cal.data}{cal.data}}
for the specific labels (default= Use all).

\item[\code{wood}] Optional, character vector of wood type specific calibration
data that should be used (one of \code{c("Diffuse-porous", "Ring-porous", "Coniferous")}).
See \code{\LinkA{cal.data}{cal.data}} for the specific labels (default= Use all).

\item[\code{calib}] Optional \code{data.frame} containing raw calibration experiment values.
Required columns include: \code{[ ,1]} \eqn{K = K}{} values measured with the probe (numeric),
and [,2] \eqn{SFD =}{} Gravimetrically measured sap flux density (\eqn{cm^3 cm^{-2} h^{-1}}{})) (numeric).
If not provided, literature values are used.

\item[\code{a}] Optional, numeric value for the calibration curve (\eqn{SFD = aK^b}{}).
No uncertainty can be calculated when this value is provided.

\item[\code{b}] Optional, numeric value for the calibration curve (\eqn{SFD = aK^b}{}).
No uncertainty can be calculated when this value is provided.

\item[\code{decimals}] Integer, the number of decimals of the output (default = 10).

\item[\code{make.plot}] Logical; if \code{TRUE}, a plot is generated showing
the calibration curve with \eqn{K vs sap flux density}{} (\eqn{cm^3 cm^{-2} h^{-1}}{})).

\item[\code{df}] Logical; If \code{TRUE}, output is provided in a \code{data.frame} format
with a timestamp and a value column. If \code{FALSE}, output
is provided as a \code{zoo} vector object (default = FALSE).
\end{ldescription}
\end{Arguments}
%
\begin{Details}\relax
The function fits a calibration curve (\eqn{SFD = aK^b}{})
through all selected raw calibration data. If multiple studies are provided,
multiple calibration curves are fitted. In case a single calibration dataset
is provided a bootstrap resampling is applied (n = 100) to determined the
mean and 95\% confidence interval of the fit. When multiple calibration curves
are requested the mean and 95\% confidence interval is determined on the fitted functions.
The mean and confidence interval are used to calculate \eqn{SFD}{} from \eqn{K}{}.
\end{Details}
%
\begin{Value}
A list containing either a \code{zoo} object or \code{data.frame} in the appropriate format
for other functionalities (see \code{\LinkA{tdm\_dt.max}{tdm.Rul.dt.max}} output specifications), as well as
all \eqn{SFD}{} values for each method are provided and added to the
\code{\LinkA{is.trex}{is.trex}}-compliant object (e.g., [['sfd.pd']], [['sfd.mw']])
if this format was provided as an input, and,
finally, a \code{data.frame} is provided with the mean and 95\% confidence
interval of the applied calibration functions (see [['model.ens']]).
If an individual time series is provided for input with \eqn{K}{} values an alternative output is provided:

\begin{description}

\item[input] K values provided as input.
\item[sfd.input] \eqn{SFD}{} values calculated for the input according to the mean of the calibration function.
\item[model.ens] A \code{data.frame} providing the mean and 95\% confidence interval of the applied calibration function.
\item[out.param] A \code{data.frame} with the coefficients of calibration function.

\end{description}

\end{Value}
%
\begin{References}\relax
Peters RL, Fonti P, Frank DC, Poyatos R, Pappas C, Kahmen A, Carraro V, Prendin AL, Schneider L, Baltzer JL,
Baron-Gafford GA, Dietrich L, Heinrich I, Minor RL, Sonnentag O, Matheny AM, Wightman MG, Steppe K. 2018.
Quantification of uncertainties in conifer sap flow measured with the thermal dissipation method.
New Phytologist 219:1283-1299 <doi: 10.1111/nph.15241>

Flo V, Martinez-Vilalta J, Steppe K, Schuldt B, Poyatos, R. 2019.
A synthesis of bias and uncertainty in sap flow methods.
Agricultural and Forest Meteorology 271:362-374 \url{doi: 10.1016/j.agrformet.2019.03.012}
\end{References}
%
\begin{Examples}
\begin{ExampleCode}
#calculating sap flux density
## Not run: 
raw   <-is.trex(example.data(type="doy"),
    tz="GMT",time.format="%H:%M",
    solar.time=TRUE,long.deg=7.7459,
    ref.add=FALSE)

input <-dt.steps(input=raw,start="2014-05-08 00:00",
end="2014-07-25 00:50",
     time.int=15,max.gap=60,decimals=10,df=FALSE)

input[which(input<0.2)]<-NA

input <-tdm_dt.max(input, methods=c("pd","mw","dr"),
     det.pd=TRUE,interpolate=FALSE,max.days=10,df=FALSE)

output.data<-tdm_cal.sfd(input,make.plot=TRUE,df=FALSE,
wood="Coniferous")


str(output.data)
plot(output.data$sfd.pd$sfd,ylim=c(0,10))
lines(output.data$sfd.pd$q025,lty=1,col="grey")
lines(output.data$sfd.pd$q975,lty=1,col="grey")
lines(output.data$sfd.pd$sfd)

output.data$out.param

## End(Not run)

\end{ExampleCode}
\end{Examples}
\inputencoding{utf8}
\HeaderA{tdm\_damp}{Signal dampening correction}{tdm.Rul.damp}
%
\begin{Description}\relax
When long-term \eqn{K}{} time series (\textasciitilde{}3 years) are provided, one can perform
a signal dampening correction (when sensors were not re-installed;
see Peters \emph{et al.} 2018). Applying the signal dampening
correction requires  visually inspecting the correction curve
(see \code{make.plot = TRUE}). The correction curve is constructed with
the day since installation and the day of year (DOY) to account for seasonal changes in \eqn{K}{}
values. The function returns corrected \eqn{K}{} values and the applied correction curve.
\end{Description}
%
\begin{Usage}
\begin{verbatim}
tdm_damp(input, k.threshold = 0.05, make.plot = TRUE, df = FALSE)
\end{verbatim}
\end{Usage}
%
\begin{Arguments}
\begin{ldescription}
\item[\code{input}] An \code{\LinkA{is.trex}{is.trex}}-compliant object (\code{zoo} vector, \code{data.frame}) of \eqn{K}{} values containing
a timestamp and a value column.

\item[\code{k.threshold}] Numeric, the threshold below which daily maximum \eqn{K}{} values should not be considered (default = 0.05).

\item[\code{make.plot}] Logical; if \code{TRUE}, a plot is generated presenting the correction curve and the \eqn{K}{} time series.

\item[\code{df}] Logical; If \code{TRUE}, output is provided in a \code{data.frame} format
with a timestamp and a value column. If \code{FALSE}, output
is provided as a \code{zoo} vector object (default = FALSE).


@details The function fits a correction curve for signal dampening (e.g., due to wounding)
according to Peters \emph{et al.} (2018). A sensor specific function is fitted to daily maximum
\eqn{K}{} values (considering a minimum cut-off threshold; see \code{k.threshold}). Dependent variables
for the function include seasonality (DOY) and days since installation (\eqn{t}{}).
First, seasonal effects are removed by correcting the \eqn{K}{} series (residuals; \eqn{Kresid}{})
to a second-order polynomial with DOY. These residuals are then used within a
non-linear model:
\deqn{K_{resid} = (a + b * t)/(1 + c * t + d * t^{2})}{}

The fitted parameters for \eqn{t}{} (with \eqn{a}{}, \eqn{b}{}, \eqn{c}{} and \eqn{d}{}) are used to
correct \eqn{K}{} and scale it to the maximum within the first year of installation.
\strong{Note, that the stability of the fit has to be visually inspected before using the output data}.
\end{ldescription}
\end{Arguments}
%
\begin{Value}
A \code{zoo} object or \code{data.frame} in the appropriate
format for other functionalities.
See \code{\LinkA{tdm\_dt.max}{tdm.Rul.dt.max}} output specifications.
All \eqn{K}{} values for each method are provided when an \code{\LinkA{is.trex}{is.trex}}-object was used as input.
If an individual time series was provided for input with \eqn{K}{} values an alternative output is given:

\begin{description}

\item[k.cor] corrected \eqn{K}{} values according to the correction curve.
\item[k] \eqn{K}{} values provided as input.
\item[damp.mod] \code{data.frame} with the coefficients of the correction curve.


\end{description}

\end{Value}
%
\begin{References}\relax
Peters RL, Fonti P, Frank DC, Poyatos R, Pappas C, Kahmen A, Carraro V,
Prendin AL, Schneider L, Baltzer JL, Baron-Gafford GA, Dietrich L, Heinrich I,
Minor RL, Sonnentag O, Matheny AM, Wightman MG, Steppe K. 2018.
Quantification of uncertainties in conifer sap flow measured with the thermal
dissipation method. New Phytologist 219:1283-1299 \url{doi: 10.1111/nph.15241}
\end{References}
%
\begin{Examples}
\begin{ExampleCode}
## Not run: 
 #correct for dampening of the signal
raw   <-
  is.trex(
    example.data(type = "doy"),
    tz = "GMT",
    time.format = "%H:%M",
    solar.time = TRUE,
    long.deg = 7.7459,
    ref.add = FALSE
  )
input <-
  dt.steps(
    input = raw,
    time.int = 15,
    max.gap = 60,
    decimals = 6,
    df = FALSE
  )
input[which(input < 0.2)] <- NA
input <-
  tdm_dt.max(
    input,
    methods = c("pd", "mw", "dr"),
    det.pd = TRUE,
    interpolate = FALSE,
    max.days = 10,
    df = FALSE
  )
output.data <- tdm_damp(input,
                    k.threshold = 0.05,
                    make.plot = TRUE,
                    df = FALSE)
str(output.data)
head(output.data[["k.dr"]])
plot(output.data[["k.dr"]], ylab = expression(italic("K")))

## End(Not run)

\end{ExampleCode}
\end{Examples}
\inputencoding{utf8}
\HeaderA{tdm\_dt.max}{Calculate zero-flow conditions}{tdm.Rul.dt.max}
%
\begin{Description}\relax
Determine zero flow conditions (\eqn{\Delta T_{max}}{}; or \eqn{\Delta V_{max}}{})
according to four methods; namely,
1) predawn (\code{pd}),
2) moving-window (\code{mw}),
3) double regression (\code{dr}),
and 4) environmental-dependent (\code{ed}) as applied in Peters \emph{et al.} 2018.
The function can provide (\eqn{\Delta T_{max}}{} values and subsequent \emph{K} values for all methods.
All outputs are provided in a \code{list} including the input data and calculated outputs.
\end{Description}
%
\begin{Usage}
\begin{verbatim}
tdm_dt.max(input, methods = c("pd","mw","dr"),
zero.end = 8*60,
zero.start =  1*60,
interpolate = FALSE, det.pd = TRUE,
max.days = 7,
ed.window = 2*60,
vpd.input,
sr.input,
sel.max,
criteria = c(sr = 30, vpd = 0.1, cv = 0.5),
df = FALSE)
\end{verbatim}
\end{Usage}
%
\begin{Arguments}
\begin{ldescription}
\item[\code{input}] An \code{\LinkA{is.trex}{is.trex}}-compliant object of \eqn{K}{} values containing
a timestamp and a value column.

\item[\code{methods}] Character vector of the requested \eqn{\Delta T_{max}}{} methods.
Options include \code{“pd”} (predawn), \code{“mw”} (moving-window), \code{“dr”} (double regression),
and \code{“ed”} (environmental-dependent; default= \code{c(“pd”, “mw”, “dr”)}).

\item[\code{zero.end}] Numeric, optionally defines the end of the predawn period.
Values should be in minutes (e.g. predawn conditions until 08:00 = 8 * 60).
When not provided, the algorithm will automatically analyse the cyclic behaviour
of the data and define the day length.

\item[\code{zero.start}] Numeric, optionally defines the beginning of the predawn period.
Values should be in minutes (e.g., 01:00 = 1*60).

\item[\code{interpolate}] Logical: if \code{TRUE}, detected \eqn{\Delta T_{max}}{} values are linearly
interpolated. If \code{FALSE}, constant \eqn{\Delta T_{max}}{} values will be selected daily
(default = \code{FALSE}).

\item[\code{det.pd}] Logical; if \code{TRUE} and no \code{zero.end} and \code{zero.start} values are provided,
predawn \eqn{\Delta T_{max}}{} will be determined based on cyclic behaviour of the entire
time-series (default = \code{TRUE}).

\item[\code{max.days}] Numeric, defines the number of days which the \code{mw} and \code{dr}
methods will consider for their moving windows.

\item[\code{ed.window}] Numeric, defines the length of the period considered for assessing the
environmental conditions and stable \eqn{\Delta T_{max}}{} values.

\item[\code{vpd.input}] An \code{\LinkA{is.trex}{is.trex}}-compliant object (\code{zoo} time-series or \code{data.frame})
with a timestamp and a value column containing the vapour pressure deficit (\emph{vpd}; in kPa)
with the same temporal extent and time steps as the input data.

\item[\code{sr.input}] An \code{\LinkA{is.trex}{is.trex}}-compliant object (\code{zoo} time-series or \code{data.frame})
with a timestamp and a value column the solar radiation data (\emph{sr}; e.g., global radiation or \emph{PAR}).

\item[\code{sel.max}] Optional \code{zoo} time-series or \code{data.frame} with the specified \eqn{\Delta T_{max}}{}.
This option is included to change predawn \eqn{\Delta T_{max}}{} values selected with the \code{ed} method.

\item[\code{criteria}] Numeric vector, thresholds for the \code{ed} method.
Thresholds should be provided for all environmental data included in the function
(e.g. \code{c(sr = 30, vpd = 0.1)}; coefficient of variation, \emph{cv} = 0.5)

\item[\code{df}] Logical; if \code{TRUE}, output is provided in a \code{data.frame}
format with a timestamp and a value (\eqn{\Delta T}{} or \eqn{\Delta V}{}) column.
If \code{FALSE}, output is provided as a \code{zoo} object (default = \code{FALSE}).
\end{ldescription}
\end{Arguments}
%
\begin{Details}\relax
There are a variety of methods which can be applied to determine zero-flow conditions.
Zero-flow conditions are required to calculate \eqn{K = (\Delta T_{max} - \Delta T) / \Delta T}{}.
A detailed description on the methods is provided by Peters \emph{et al.} (2018).
In short, the \code{pd} method entails the selection of daily maxima occurring prior to sunrise.
This method assumes that during each night zero-flow conditions are obtained.
The algorithm either requires specific times within which it searches for a maximum,
or it analyses the cyclic pattern within the data and defines this time window.
The \code{mw} method uses these predawn \eqn{\Delta T_{max}}{} values
and calculates the maximum over a multi-day moving time-window (e.g., 7 days).
The \code{dr} methods is applied by calculating the mean over predawn \eqn{\Delta T_{max}}{}
with a specified multi-day window, removing all values below the mean,
and calculating a second mean over the same multi-day window and uses these values as \eqn{\Delta T_{max}}{}.
The \code{ed} method selects predawn \eqn{\Delta T_{max}}{} values based upon 2-hour averaged environmental
conditions prior to the detected time for the predawn \eqn{\Delta T_{max}}{}.
These environmental conditions include low vapour pressure deficit (in \eqn{kPa}{}) and low solar irradiance
(e.g., in W m-2). In addition, the coefficient of variation (cv) of predawn \eqn{\Delta T_{max}}{} are scanned for low values to
ensure the selection of stable zero-flow conditions.
\end{Details}
%
\begin{Value}
A named \code{list} of \code{zoo} time series or \code{data.frame}
objects in the appropriate format for further processing.
List items include:
\begin{description}

\item[max.pd] \eqn{\Delta T_{max}}{} time series as determined by the \code{pd} method.
\item[max.mw] \eqn{\Delta T_{max}}{} time series as determined by the \code{mw} method.
\item[max.dr] \eqn{\Delta T_{max}}{} time series as determined by the \code{dr} method.
\item[max.ed] \eqn{\Delta T_{max}}{} time series as determined by the \code{ed} method.
\item[daily\_max.pd] daily predawn \eqn{\Delta T_{max}}{} as determined by \code{pd}.
\item[daily\_max.mw] daily predawn \eqn{\Delta T_{max}}{} as determined by \code{mw}.
\item[daily\_max.dr] daily predawn \eqn{\Delta T_{max}}{} as determined by \code{dr}.
\item[daily\_max.ed] daily predawn \eqn{\Delta T_{max}}{} as determined by \code{ed}.
\item[all.pd] exact predawn \eqn{\Delta T_{max}}{} values detected with \code{pd}.
\item[all.ed] exact predawn \eqn{\Delta T_{max}}{} values detected with \code{ed}.
\item[input] \eqn{\Delta T}{} input data.
\item[ed.criteria] \code{data.frame} of the applied environmental and variability criteria used within \code{ed}.
\item[methods] \code{data.frame} of applied methods to detect \eqn{\Delta T_{max}}{}.
\item[k.pd] \eqn{K}{} values calculated by using the \code{pd} method.
\item[k.mw] \eqn{K}{} values calculated by using the \code{mw} method.
\item[k.dr] \eqn{K}{} values calculated by using the \code{dr} method.
\item[k.ed] \eqn{K}{} values calculated by using the \code{ed} method.

\end{description}

\end{Value}
%
\begin{Examples}
\begin{ExampleCode}
## Not run: 
#perform Delta Tmax calculations
raw <- is.trex(example.data(type = "doy"),
     tz = "GMT", time.format = "%H:%M", solar.time = TRUE,
     long.deg = 7.7459, ref.add = FALSE)
input <- dt.steps(input = raw, start = "2014-05-08 00:00",
         end = "2014-07-25 00:50", time.int = 15, max.gap = 60,
         decimals = 6, df = FALSE)
input[which(input<0.2)]<- NA
output.max <- tdm_dt.max(input, methods = c("pd", "mw", "dr"),
                 det.pd = TRUE, interpolate = FALSE,
                 max.days = 10, df = FALSE)

str(output.max)

plot(output.max$input, ylab = expression(Delta*italic("V")))

lines(output.max$max.pd, col = "green")
lines(output.max$max.mw, col = "blue")
lines(output.max$max.dr, col = "orange")

points(output.max$all.pd, col = "green", pch = 16)

legend("bottomright", c("raw", "max.pd", "max.mw", "max.dr"),
        lty = 1, col = c("black", "green", "blue", "orange") )


## End(Not run)
\end{ExampleCode}
\end{Examples}
\inputencoding{utf8}
\HeaderA{tdm\_hw.cor}{Heartwood correction}{tdm.Rul.hw.cor}
%
\begin{Description}\relax
The function corrects for the proportion of the probe that is installed
within the non-conductive heartwood according to Clearwater \emph{et al.} (1999).
The function requires \eqn{\Delta T_{max}}{}, the probe length and the
sapwood thickness. The correction is applied on the \eqn{\Delta T}{} (or \eqn{\Delta V}{}) values and \eqn{K}{}
is recalculated accordingly. When an \code{\LinkA{is.trex}{is.trex}}-compliant object is
provided, the \eqn{K}{} values for each method are determined (see \code{\LinkA{tdm\_dt.max}{tdm.Rul.dt.max}}.
\end{Description}
%
\begin{Usage}
\begin{verbatim}
tdm_hw.cor (input, dt.max, probe.length = 20,
            sapwood.thickness = 18, df = FALSE)
\end{verbatim}
\end{Usage}
%
\begin{Arguments}
\begin{ldescription}
\item[\code{input}] A \code{\LinkA{tdm\_dt.max}{tdm.Rul.dt.max}} ouput or \code{\LinkA{is.trex}{is.trex}}-compliant object of \eqn{\Delta T }{} (or \eqn{\Delta V}{}) values containing
a timestamp and a value column.

\item[\code{dt.max}] Optional \code{zoo} object or \code{data.frame} (columns = “timestamp” or “value”)
containing the \eqn{\Delta T_{max}}{} when no \code{\LinkA{is.trex}{is.trex}}-compliant object is provided.

\item[\code{probe.length}] Numeric, the length of the TDM probes in mm.

\item[\code{sapwood.thickness}] Numeric, the sapwood thickness in mm.

\item[\code{df}] Logical; If \code{TRUE}, output is provided in a \code{data.frame} format
with a timestamp and a value column. If \code{FALSE}, output
is provided as a \code{zoo} vector object (default = \code{FALSE}).
\end{ldescription}
\end{Arguments}
%
\begin{Details}\relax
The function applied the correction provided by Clearwater \emph{et al.} 1999.
\eqn{\Delta T}{} (or \eqn{\Delta V}{}) was corrected (denoted as \eqn{\Delta T_{sw}}{}) for the proportion of
the probe that was inserted into the conducting sapwood vs the
proportion of the probe that was inserted into the nonconductive heartwood
(\eqn{\gamma}{} in mm mm-1). Together with \eqn{\Delta T_{max}}{}, \eqn{\Delta T}{} was corrected
according to the following equation:
\deqn{\Delta T_{sw} = (\Delta T~-~(1~–~\gamma) *  \Delta T_{max}) / \gamma}{}

\eqn{\Delta T_{sw}}{} together with \eqn{\Delta T_{max}}{} is then recalculated to \eqn{K}{}.
\end{Details}
%
\begin{Value}
A \code{zoo} object or \code{data.frame} in the appropriate
format for other functionalities. See \code{\LinkA{tdm\_dt.max}{tdm.Rul.dt.max}} for output specifications.
All \eqn{K}{} values for each method are provided when an output
from \code{\LinkA{tdm\_dt.max}{tdm.Rul.dt.max}} was provided.
If individual time series are provided for \code{input} and \code{tdm\_dt.max} an alternative output is provided:

\begin{description}


\item[input] = \eqn{\Delta T}{} input data.
\item[dt.max]  \eqn{\Delta T_{max}{\Delta Tmax}}{} input data.
\item[dtsw] Corrected \eqn{\Delta T}{} data.
\item[k.value] \eqn{K}{} values calculated according to Clearwater et al. (1999).
\item[settings] data.frame of the applied \code{probe.length} and \code{sapwood.thickness}


\end{description}

\end{Value}
%
\begin{References}\relax
Clearwater MJ, Meinzer FC, Andrade JL, Goldstein G, Holbrook NM. 1999.
Potential errors in measurement of nonuniform sap flow using heat dissipation probes.
Tree Physiology 19:681–687 \url{doi: 10.1093/treephys/19.10.681}
\end{References}
%
\begin{Examples}
\begin{ExampleCode}
## Not run: 
#correct for probes being inserted into the heartwood
raw   <-is.trex(example.data(type="doy"),
          tz="GMT",time.format="%H:%M",solar.time=TRUE,
          long.deg=7.7459,ref.add=FALSE)
input <- dt.steps(input=raw,
                   start="2014-05-08 00:00",
                   end="2014-07-25 00:50",
                  time.int=15,max.gap=60,decimals=6,df=F)
input[which(input<0.2)]<-NA
input <-tdm_dt.max(input, methods=c("pd","mw","dr"),
                 det.pd=TRUE,interpolate=FALSE,max.days=10,df=FALSE)
output.data<-tdm_hw.cor(input,probe.length=20,
                   sapwood.thickness=18,df=FALSE)
plot(output.data$k.dr,col="orange")
lines(input$k.dr)

## End(Not run)

\end{ExampleCode}
\end{Examples}
\inputencoding{utf8}
\HeaderA{tdm\_uncertain}{Uncertainty and sensitivity analysis}{tdm.Rul.uncertain}
%
\begin{Description}\relax
Quantifies the induced uncertainty on \eqn{SFD}{} and \eqn{K}{} time series due to the variability
in input parameters applied during TDM data processing. Moreover, it applies a global sensitivity
analysis to quantify the impact of each individual parameter on three relevant outputs derived from \eqn{SFD}{} and \eqn{K}{}, namely:
i) the mean daily sum of water use,
ii) the variability of maximum daily \eqn{SFD}{} or \eqn{K}{} values,
iii) and the duration of daily sap flow.
This function provides both the uncertainty and sensitivity indices, as time-series of \eqn{SFD}{} and \eqn{K}{} with the mean,
standard deviation (\eqn{sd}{}) and confidence interval (CI) due to parameter uncertainty.
\strong{Users should ensure that no gaps are present within the input data and environmental time series}.
\end{Description}
%
\begin{Usage}
\begin{verbatim}
tdm_uncertain(input, vpd.input, sr.input, method = "pd", n = 2000,
  zero.end = 8 * 60, range.end = 16, zero.start = 1 * 60,
  range.start = 16, probe.length = 20, sw.cor = 32.28, sw.sd = 16,
  log.a_mu = 4.085, log.a_sd = 0.628, b_mu = 1.275, b_sd = 0.262,
  max.days_min = 1, max.days_max = 7, ed.window_min = 8,
  ed.window_max = 16, criteria.vpd_min = 0.05,
  criteria.vpd_max = 0.5, criteria.sr_mean = 30,
  criteria.sr_range = 30, criteria.cv_min = 0.5, criteria.cv_max = 1,
  min.sfd = 0.5, min.k = 0, make.plot = TRUE, df = FALSE)
\end{verbatim}
\end{Usage}
%
\begin{Arguments}
\begin{ldescription}
\item[\code{input}] An \code{\LinkA{is.trex}{is.trex}}-compliant object (\code{zoo} object

\item[\code{vpd.input}] An \code{\LinkA{is.trex}{is.trex}}-compliant object (\code{zoo} object,
\code{data.frame}) containing a timestamp and a vapour pressure deficit
(\eqn{VPD}{}; in \eqn{kPa}{}) column with the same temporal extent and time steps as the \code{input} object.

\item[\code{sr.input}] An \code{\LinkA{is.trex}{is.trex}}-compliant object (\code{zoo} object,
\code{data.frame}) a timestamp and a solar radiation data (sr; e.g., global radiation or PAR)
column with the same temporal extent and time steps as the \code{input} object.

\item[\code{method}] Character, specifies the \eqn{\Delta T_{max}}{} method on which the
sensitivity and uncertainty analyse are to be performed on (see \code{\LinkA{tdm\_dt.max}{tdm.Rul.dt.max}}).
Only one method can be selected, including the pre-dawn (\code{"pd"}), moving window (\code{"mw"}),
double regression (\code{"dr"}) or the environmental dependent (\code{"ed"}) method (default = \code{"pd"}).

\item[\code{n}] Numeric, specifies the number of times the bootstrap resampling procedure is repeated (default = 2000).
Keep in mind that high values increase processing time.

\item[\code{zero.end}] Numeric, defines the end of the predawn period.
Values should be in minutes (e.g., predawn conditions until 8:00 = 8*60; default = 8*60).

\item[\code{range.end}] Numeric, defines the number of time steps for \code{zero.end} (the minimum time step of the input)
for which an integer sampling range will be defined (default = 16, assuming a 15-min resolution or a 2 hour range).

\item[\code{zero.start}] Numeric, defines the number of time steps for \code{zero.end} (the minimum time step of the input)
for which an integer sampling range will be defined (default = 16, assuming a 15-min resolution or a 2 hour range).

\item[\code{range.start}] Numeric, defines the number of time steps for \code{zero.start} (the minimum time step of the input)
for which an integer sampling range will be defined (default = 16, assuming a 15-min resolution or a 2 hour range).

\item[\code{probe.length}] Numeric, the length of the TDM probes in mm (see \code{\LinkA{tdm\_hw.cor}{tdm.Rul.hw.cor}}; default = 20 mm).

\item[\code{sw.cor}] Numeric, the sapwood thickness in mm. Default conditions assume the sapwood thickness is equal to a standard probe length (default = 20).

\item[\code{sw.sd}] Numeric, the standard deviation for sampling sapwood thickness sampling from a normal distribution (default = 16 mm;
defined with a European database on sapwood thickness measurements).

\item[\code{log.a\_mu}] Numeric, value providing the natural logarithm of the calibration parameter \eqn{a}{} (see \code{\LinkA{tdm\_cal.sfd}{tdm.Rul.cal.sfd}}; \eqn{SFD = aK^b}{}).
This value can be obtained from \code{\LinkA{tdm\_cal.sfd}{tdm.Rul.cal.sfd}} (see \code{out.param}).
Default conditions are determined by using all calibration data as described in \code{\LinkA{cal.data}{cal.data}} (default = 4.085).

\item[\code{log.a\_sd}] Numeric, the standard deviation of the \eqn{a}{} parameter (see \code{log.a\_mu}) used within the calibration curve
for calculating \eqn{SFD}{} (default = 0.628).

\item[\code{b\_mu}] Numeric, the value of the calibration parameter \eqn{b}{} (see \code{\LinkA{tdm\_cal.sfd}{tdm.Rul.cal.sfd}}; \eqn{SFD = aK^b}{}).
This value can be obtained from \code{\LinkA{tdm\_cal.sfd}{tdm.Rul.cal.sfd}} (see \code{out.param}).
Default conditions are determined by using all calibration data as described in \code{\LinkA{cal.data}{cal.data}} (default = 1.275).

\item[\code{b\_sd}] Numeric, the standard deviation of the \eqn{b}{} parameter (see \code{log.a\_mu}) used within the calibration curve
for calculating \eqn{SFD}{} (default = 0.262).

\item[\code{max.days\_min}] Numeric, the minimum value for an integer sampling range of \code{max.days}
(see \code{\LinkA{tdm\_dt.max}{tdm.Rul.dt.max}} for the \code{"mw"} and \code{"dr"} \eqn{\Delta T_{max}}{} method).
As the \code{"mw"} and \code{"dr"} method apply a rolling maximum or mean, the provided value should be an
uneven number (see \code{\LinkA{tdm\_dt.max}{tdm.Rul.dt.max}}; default = 5; required for the \code{"mw"} and \code{"dr"} \eqn{\Delta T_{max}}{} method).

\item[\code{max.days\_max}] Numeric, the maximum value for an integer sampling range of \code{max.days}
(see \code{\LinkA{tdm\_dt.max}{tdm.Rul.dt.max}} for the \code{"mw"} and \code{"dr"} \eqn{\Delta T_{max}}{} method).
As the \code{"mw"} and \code{"dr"} method apply a rolling maximum or mean, the provided value should be an
uneven number (see \code{\LinkA{tdm\_dt.max}{tdm.Rul.dt.max}}; default = 5; required for the \code{"mw"} and \code{"dr"} \eqn{\Delta T_{max}}{} method).

\item[\code{ed.window\_min}] Numeric, the minimum number of time steps for the \code{ed.window parameter} (see \code{\LinkA{tdm\_dt.max}{tdm.Rul.dt.max}}; the minimum time step of the input)
for which an integer sampling range will be defined (default = 8, assuming a 15-min resolution or a 2 hour range; required for the \code{"ed"} \eqn{\Delta T_{max}}{} method).

\item[\code{ed.window\_max}] Numeric, the maximum number of time steps for the \code{ed.window} sampling range
(default = 16, assuming a 15-min resolution or a 4 hour range; required for the \code{"ed"} \eqn{\Delta T_{max}}{} method).

\item[\code{criteria.vpd\_min}] Numeric, value in \eqn{kPa}{} defining the minimum for the fixed sampling range to define the vapour pressure deficit (VPD)
threshold to establish zero-flow conditions (default = 0.05 \eqn{kPa}{}; see \code{\LinkA{tdm\_dt.max}{tdm.Rul.dt.max}}; required for the \code{"ed"} \eqn{\Delta T_{max}}{} method).

\item[\code{criteria.vpd\_max}] Numeric, value in \eqn{kPa}{} defining the maximum for the fixed sampling range to define the VPD threshold to establish
zero-flow conditions (default = 0.5 \eqn{kPa}{}; required for the \code{"ed"}  \eqn{\Delta T_{max}}{} method).

\item[\code{criteria.sr\_mean}] Numeric value defining the mean \code{sr.input} value around which the fixed sampling
range for the solar irradiance threshold should be established for defining zero-flow conditions
(see \code{\LinkA{tdm\_dt.max}{tdm.Rul.dt.max}}; default = 30 W m-2; required for the \code{"ed"} \eqn{\Delta T_{max}}{} method).

\item[\code{criteria.sr\_range}] Numeric, the range (in \%) around \code{criteria.sr\_mean} for establishing the solar irradiance threshold
(see \code{\LinkA{tdm\_dt.max}{tdm.Rul.dt.max}}; default = 30\%; required for the \code{"ed"} \eqn{\Delta T_{max}}{} method).

\item[\code{criteria.cv\_min}] Numeric, value (in \%) defining the minimum value for the fixed sampling range to
determine the coefficient of variation (CV) threshold for establishing zero-flow conditions
(default = 0.5\%; see \code{\LinkA{tdm\_dt.max}{tdm.Rul.dt.max}}; required for the \code{"ed"} \eqn{\Delta T_{max}}{} method).

\item[\code{criteria.cv\_max}] Numeric, value (in \%) defining the maximum value for the fixed sampling range
to determine the coefficient of variation (CV) threshold for establishing zero-flow conditions
(default = 1\%; see \code{\LinkA{tdm\_dt.max}{tdm.Rul.dt.max}}; required for the \code{"ed"} \eqn{\Delta T_{max}}{} method).

\item[\code{min.sfd}] Numeric, defines at which \eqn{SFD}{} (\eqn{cm^3 cm^{-2} h^{-1}}{}) zero-flow conditions are expected.
This parameter is used to define the duration of daily sap flow based on \eqn{SFD}{} (default = \eqn{0.5 cm^3 cm^{-2} h^{-1}}{}).

\item[\code{min.k}] Numeric value defining at which \eqn{K}{} (dimensionless, -) zero-flow are expected.
This parameter is used to define the duration of daily sap flow based on \eqn{K}{} (default = 0).

\item[\code{make.plot}] Logical; If \code{TRUE}, a plot is generated presenting the sensitivity and uncertainty analyses output (default = \code{TRUE}).

\item[\code{df}] Logical; If \code{TRUE}, output is provided in a \code{data.frame} format with a timestamp and a value column.
If \code{FALSE}, output is provided as a zoo vector object (default = \code{FALSE}).
\end{ldescription}
\end{Arguments}
%
\begin{Details}\relax
Uncertainty and sensitivity analysis can be performed on TDM \eqn{\Delta T}{} (or \eqn{\Delta V}{}) measurements.
The function applies a Monte Carlo simulation approach (repetition defined by \code{n})
to determine the variability in relevant output variables (defined as uncertainty)
and quantifies the contribution of each parameter to this uncertainty (defined as sensitivity).
To generate variability in the selected input parameters a Latin Hypercube Sampling is performed with a default
or user defined range of parameter values per \eqn{\Delta T_{max}}{} method (see \code{\LinkA{tdm\_dt.max}{tdm.Rul.dt.max}()}).
The sampling algorithm generates multiple sampling distributions, including an integer sampling range (for \code{zero.start},
\code{zero.end}, \code{max.days}, and \code{ed.window}), a continuous sampling range (criteria for \code{sr}, \code{vpd} and \code{cv}),
and a normal distribution (for \code{sw.cor} and calibration parameters \code{a} and \code{b}).
Within this algorithm no within-day interpolations are made between the \eqn{\Delta T_{max}}{} points
(see \code{\LinkA{tdm\_dt.max}{tdm.Rul.dt.max}}, \code{interpolate = FALSE}). This approach ensures near-random sampling across different
types of sampling distributions, while avoiding the need for increasing the number of replicates
(which increases computation time). For the application of this approach one needs to;
i) select the output of interest,
ii) identify the relevant input parameters, and
iii) determine the parameter range and distribution.
For a given time-series three output variables are considered, calculated as the mean over the entire time-series,
to be relevant, namely;
i) mean daily sum of water use (or Sum, expressed in \eqn{cm^3 cm^{-2} d^{-1}}{} for \eqn{SFD}{} and unitless for \eqn{K}{}),
ii) the variability of maximum \eqn{SFD}{} or \eqn{K}{} values (or CV, expressed as the coefficient of variation in \%
as this alters climate response correlations), and
iii) the duration of daily sap flow based on \eqn{SFD}{} or \eqn{K}{} (or Duration, expressed in hours per day dependent on a threshold,
see \code{min.sfd} and \code{min.k}).
A minimum threshold to define zero-flow \eqn{SFD}{} or \eqn{K}{} is required for the duration calculation
as small variations in night-time \eqn{SFD}{} or \eqn{K}{} are present. All data-processing steps
(starting with \code{"tdm\_"}) are incorporated within the function, excluding \code{\LinkA{tdm\_damp}{tdm.Rul.damp}()}
due to the need for detailed visual inspection and significantly longer computation time.

For the sensitivity analysis the total overall sensitivity indices are determined according strategy originally proposed by
Sobol' (1993), considering the improvements applied within the sensitivity R package.
The method proposed by Sobol’ (1993) is a variance-based sensitivity analysis,
where sensitivity indices (dimensionless from 0 to 1) indicate the partial variance contribution
by a given parameter over the total output variance (e.g., Pappas \emph{et al.} 2013).
This global sensitivity analysis facilitates the identification of key parameters for data-processing
improvement and highlights methodological limitations. Users should keep in mind that parameter ranges represent
a very critical component of any sensitivity analysis and should be critically assessed and clearly reported
for each case and analytical purpose. Moreover, it is advised to run this function on one growing season of input data to reduce processing time.
\end{Details}
%
\begin{Value}
A named \code{list} of \code{zoo} or \code{data.frame} objects in the appropriate format for other functionalities.
Items include:

\begin{description}


\item[output.data] data.frame containing uncertainty and sensitivity indices for \eqn{SFD}{} and K and the included parameters.
This includes the mean uncertainty/sensitivity [,"mean"], standard deviation [,"sd"], upper [,"ci.min"] and lower [,"ci.max"]
95\% confidence interval.

\item[output.sfd] zoo object or data.frame with the \eqn{SFD}{} time series obtained from the bootstrap resampling.
This includes the mean uncertainty/sensitivity [,"mean"], standard deviation [,"sd"], upper [,"CIup"] and lower [,"CIlo"]
95\% confidence interval.

\item[output.k] zoo object or data.frame with the K time series obtained from the bootstrap resampling.
This includes the mean uncertainty/sensitivity [,"mean"], standard deviation [,"sd"], upper [,"ci.max"]
and lower [,"ci.min"] 95\% confidence interval. 


\item[param] a data.frame with an overview of selected parameters used within \code{\LinkA{tdm\_uncertain}{tdm.Rul.uncertain}()} function. 



\end{description}

\end{Value}
%
\begin{References}\relax
Sobol’ I. 1993. Sensitivity analysis for nonlinear mathematical models.
Math. Model Comput. Exp. 1:407-414

Pappas C, Fatichi S, Leuzinger S, Wolf A, Burlando P. 2013.
Sensitivity analysis of a process-based ecosystem model: Pinpointing parameterization and structural issues.
Journal of Geophysical Research 118:505-528 \url{doi: 10.1002/jgrg.20035}
\end{References}
%
\begin{Examples}
\begin{ExampleCode}
## Not run: 
#perform an uncertainty and sensitivity analysis on "dr" data processing
raw   <- example.data(type="doy")
input <- is.trex(raw, tz="GMT", time.format="%H:%M",
           solar.time=TRUE, long.deg=7.7459, ref.add=FALSE, df=FALSE)
input<-dt.steps(input,time.int=15,start="2013-04-01 00:00",
             end="2013-11-01 00:00",max.gap=180,decimals=15)
output<- tdm_uncertain(input, probe.length=20, method="pd",
               n=2000,sw.cor=32.28,sw.sd=16,log.a_mu=3.792436,
               log.a_sd=0.4448937,b_mu=1.177099,b_sd=0.3083603,
               make.plot=TRUE)

## End(Not run)

\end{ExampleCode}
\end{Examples}
\inputencoding{utf8}
\HeaderA{TREX}{TREX: TRree sap flow EXtractor.}{TREX}
\aliasA{TREX-package}{TREX}{TREX.Rdash.package}
%
\begin{Description}\relax
Performs data assimilation, processing and analyses on sap flow data obtained with the thermal dissipation method (TDM).
The package includes functions for gap filling time-series data, detecting outliers, calculating data-processing uncertainties and,
generating uniform data output and visualisation. The package is designed to deal with large quantities of data and apply commonly
used data-processing methods. The functions have been validated on data collected from different tree species across the northern hemisphere
(Peters et al. 2018 <doi: 10.1111/nph.15241>).
\end{Description}
\inputencoding{utf8}
\HeaderA{vpd}{Vapor pressure deficit measurements (raw)}{vpd}
\keyword{datasets}{vpd}
%
\begin{Description}\relax
Returns an example dataset of vapour pressure deficit (\eqn{VPD}{}) monitoring from
2012-2015 at 1300 m a.s.l. in the Swiss Alps (Loetschental, Switzerland; Peters \emph{et al.} 2019).
Sensors were installed at the site on a central tower (\textasciitilde{}2.5 m above the ground)
within the canopy to measure air temperature and relative humidity (Onset, USA, U23-002Pro) with a 15‐min resolution.
Vpd (\eqn{kPa}{} was calculated from the air temperature and relative humidity measurements according to WMO (2008).
\end{Description}
%
\begin{Usage}
\begin{verbatim}
vpd
\end{verbatim}
\end{Usage}
%
\begin{Format}
Provides an \code{\LinkA{is.trex}{is.trex}}-compliant object with 135840 rows and 1 column.

\begin{description}

\item[index] Date of the measurements in solar time (“yyyy-mm-dd”) (\code{character})
\item[value] kPa values obtained from the site-specific monitoring (\code{numeric})

\end{description}
\end{Format}
%
\begin{References}\relax
Peters RL, Speich M, Pappas C, Kahmen A, von Arx G, Graf Pannatier E, Steppe K, Treydte K, Stritih A, Fonti P. 2018.
Contrasting stomatal sensitivity to temperature and soil drought in mature alpine conifers.
Plant, Cell \& Environment 42:1674-1689 \url{doi: 10.1111/pce.13500}

WMO. 2008.Guide to meteorological instruments and methods of observation, appendix 4B, WMO‐No. 8 (CIMO Guide).
Geneva, Switzerland: World Meteorological Organization.
\end{References}
\printindex{}
\end{document}
